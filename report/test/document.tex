\documentclass{jsarticle}
% \usepackage[here]

\makeatletter
\newcommand{\figcaption}[1]{\def\@captype{figure}\caption{#1}}
\newcommand{\tblcaption}[1]{\def\@captype{table}\caption{#1}}
\makeatother

\usepackage{float}
\usepackage[dvipdfmx]{graphicx}


\begin{document}

\title{情報工学実験}
\author{hogehoge}
\date{2015年4月26日}
\maketitle

\section{目的}

アナログ通信の中で代表的なAMとFMの通信システムを構成し、各部の動作を確認し、特性を測定し、その原理を理解する。

\section{理論}

ここは理論だよ!!!書こうね!!!

\section{実験方法}

\begin{figure}[H]
\begin{center}
  \begin{tabular}{c}

    % 1枚目の画像
    \begin{minipage}{0.5\hsize}
      \begin{center}
        \includegraphics[clip, width=60mm]{original.png}
        \hspace{1.6cm} (a)1枚目
      \end{center}
    \end{minipage}

    % 2枚目の画像
    \begin{minipage}{0.5\hsize}
      \begin{center}
        \includegraphics[clip, width=60mm]{original.png}
        \hspace{1.6cm} (b)2枚目
      \end{center}
    \end{minipage}

  \end{tabular}
  \caption{画像}
  \label{fig:img}
  \end{center}
\end{figure}

\begin{itemize}
  \item その1
  \item その2
\end{itemize}



\begin{enumerate}
  \item てす
  \item ててす
  \item てててす
\end{enumerate}



\section{結果}

\subsection{AMの実験}

\subsection{2}


ここは結果をかくよ!!
\begin{figure}[H]
\begin{center}
  \begin{tabular}{c}

    % 1枚目の画像
    \begin{minipage}{0.5\hsize}
      \begin{center}
        \includegraphics[clip, width=60mm]{original.png}
        \hspace{1.6cm} (a)1枚目
      \end{center}
    \end{minipage}

    % 2枚目の画像
    \begin{minipage}{0.5\hsize}
      \begin{center}
        \includegraphics[clip, width=60mm]{original.png}
        \hspace{1.6cm} (b)2枚目
      \end{center}
    \end{minipage}

  \end{tabular}
  \caption{画像}
  \label{fig:img}
  \end{center}
\end{figure}


\begin{figure}[h]
\begin{tabular}{ccc}
  \begin{minipage}{.30\textwidth}
  	\begin{tabular}{lc}
	論理演算	& AND \\
	論理式	& $F=A\cdot B$\\
	真理値表	& 表\ref{table:and}\\
	図記号	& 図\ref{fig:and}
	\end{tabular}

  \end{minipage}
  \begin{minipage}{.30\textwidth}

	\centering
	\includegraphics[width=3cm]{original.png}
    \figcaption{ANDの図記号}
    \label{fig:and}
  \end{minipage}

  \begin{minipage}{.30\textwidth}
    \begin{center}
    \tblcaption{ANDの真理値表}
    \label{table:and}
	\begin{tabular}{|c|c||c|}\hline
		A & B & F \\ \hline\hline
		0 & 0 & 0 \\ \hline
		0 & 1 & 0 \\ \hline
		1 & 0 & 0 \\ \hline
		1 & 1 & 1 \\ \hline
	\end{tabular}
    \end{center}
  \end{minipage}
\end{tabular}
\end{figure}

写真を貼ろうね!!!!



\section{考察}

考察も早くしよう!!!!!

\begin{figure}[H]
\begin{center}
  \begin{tabular}{c}

    % 1枚目の画像
    \begin{minipage}{0.5\hsize}
      \begin{center}
        \includegraphics[clip, width=60mm]{original.png}
        \hspace{1.6cm} (a)1枚目
      \end{center}
    \end{minipage}

    % 2枚目の画像
    \begin{minipage}{0.5\hsize}
      \begin{center}
        \includegraphics[clip, width=60mm]{original.png}
        \hspace{1.6cm} (b)2枚目
      \end{center}
    \end{minipage}

  \end{tabular}
  \caption{画像}
  \label{fig:img}
  \end{center}
\end{figure}

\section{ラジオ}


レポートやだポヨーーーーー

\end{document}

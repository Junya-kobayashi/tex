\documentclass[11pt,a4j]{jsarticle}

\usepackage{float,array,booktabs,here}
\usepackage{amsmath}
\usepackage[dvipdfmx]{graphicx}
\usepackage[top=20truemm,bottom=25truemm,left=20truemm,right=20truemm]{geometry}
\usepackage{url}
\usepackage{listings, jlisting}

\lstset{language=c,
  frame=single,
  stepnumber=1,
  numbersep=2pt,
  tabsize=2,
  basicstyle=\verysmall\ttfamily,
  stringstyle=\small\texttt,
  commentstyle=\slshape,
  captionpos=b,
  columns=[l]{fullflexible}
}

\makeatletter
\newcommand{\figcaption}[1]{\def\@captype{figure}\caption{#1}}
\newcommand{\tblcaption}[1]{\def\@captype{table}\caption{#1}}
\makeatother

\newcommand{\Maru}[1]{\ooalign{
\ifnum#1<10 \hfil\resizebox{.9\width}{.85\height}{#1}\hfil
\else
\hfil\resizebox{.6\width}{.8\height}{#1}\hfil
\fi
\crcr
\raise.1ex\hbox{$\bigcirc$}}}



\begin{document}

\input{title}


\section{課題1}
設問2, 4を選択して考察を加える。

\subsection{設問2}
光学式モーションキャプチャ装置と慣性センサ(IMU)式モーションキャプチャ装置について計測原理を調査し、
「遮蔽物の少ないスタジオ内のセットにおける演技者の全身運動の記録」と
「走行中のバイクにおける運転者の全身運動の記録」という2つのケースを想定して、
計測の精度・設置性・遮蔽に対する頑健性などの観点を踏まえて利点と欠点を考察せよ.

\subsubsection{光学式モーションキャプチャ}
まず、モーションキャプチャとは、人や物の動きをデジタル化するシステムのことである。
モーションキャプチャを行うためのカメラには、赤外線を発光するストロボライトが内蔵されている。
そこでマーカーが、ストロボの光を反射することで、カメラがマーカーの位置を認識する。

このカメラで捉えた位置は2次元の位置であるが、実際に使用するには三次元の座標が必要である。
そこで、最初に``キャリブレーション''により、複数台のカメラの互いの位置と角度を定義し
キャリブレーション情報と各カメラの2次元座標の情報を組み合わせることにより、3次元座標が算出することができる。

この光学式モーションキャプチャの制度は以下のような項目によって左右される。

\begin{itemize}
  \item カメラの分解能
  \item レンズの歪み
  \item キャリブレーションの精度
  \item マーカーの形状
  \item カメラの配置
\end{itemize}

\subsubsection{慣性センサ式モーションキャプチャ}
全身にジャイロ・加速度・地磁気センサを搭載したスーツを着てモーションキャプチャを行う。
その各センサの値を利用して各部位の姿勢、加速度などをデータ化することでモーションキャプチャが行われる。

\subsubsection{``遮蔽物の少ないスタジオ内のセットにおける演技者の全身運動の記録''について}

スタジオ内のセットという特殊な環境では、光学式モーションキャプチャの計測装置を設置して、キャリブレーションまで
行うことが可能である。光学式モーションキャプチャでは一旦設置が完了した場所においては正確な位置検出を行うことが可能であるため。

よって、計測の精度、設置性、遮蔽に対する頑健性という3つの点に対して、光学式モーションキャプチャで
満たすことができるので、光学式モーションキャプチャが適切なケースと考えられる。

\subsubsection{``走行中のバイクにおける運転者の全身運動の記録''について}

走行中という、設置が必要とされる光学式モーションキャプチャでのキャプチャは不可能である。
よって設置性、遮蔽への頑健性などの点から、慣性センサ式モーションキャプチャが適切なケースであると考えられる。




\subsection{設問4}
Oculus Riftなどのウェアラブルな情報端末が普及した場合,どのような社会的な問題が考えられるか.
没入型バーチャルリアリティのシステムにおけるコンテンツについて例を挙げながら,それを踏まえた考察を与えよ

社会的な問題として考えられることは3つある。

一つ目は、没入型コンテンツの世界に依存しすぎるということである。
具体的には、今のゲームをし過ぎてしまう子供達の問題と同じようなことである。
依存を防ぐためには、利用者自身、またその保護者などが適切に使用をコントロールできる仕組みを作ることが必要と考えられる。


二つ目は、VR空間に長時間いることで現れる健康被害である。

VR空間に25時間いるという実験が行われたが、その動画(\cite{youtube1}より)を見ると以下のようなことがわかる。
また、動画とともに、GIGAZINEのホームページ\cite{gigazine}も参考にした。
健康的な被害について、以下の表\ref{tab:vrex}にまとめた。

\begin{table}[H]
  \caption{被験者の様子}
  \label{tab:vrex}
  \small
  \begin{center}
      \begin{tabular}{c|c|c}
        \hline
        \toprule
        経過時間	&	コメント	&	確認できる症状	\\
        \midrule
        0時間	&	準備万端	&	特になし	\\
        1時間	&	時の流れが遅く感じる	&	軽度の疲労 \\
        9時間	&	気分が悪い	&	重度の疲労	\\
        17時間	&	吐きそうだ…バケツをくれ	&	吐き気、嘔吐 \\
        21時間	&	俺はどこだ…俺はどこにいるかわからない	&	錯乱状態 \\
        \hline
      \end{tabular}
  \end{center}
\end{table}

1時間程度というそこまで長くない時間で違和感を覚えている。これは、コンテンツの不足からくる、飽きであると考えられる。

この実験はかなりストイックな実験であるので、症状としては強めに出ていると考えられるが、それでも連続してVR空間にいるというのは
強くストレスが溜まることであると考えられる。

一つの原因としては、頭にHMDという重みのあるものを装着していることが考えれらる。
そこからくる肩への疲労というのはストレスの大きな要因であろう。

よって、HMDを装着することでの疲れ、また自分のアイデンティティが見失いそうになる
VR空間というのは、さらに研究を進めていくことが必要になると思う。

3つ目は、少子化などの問題が発生しうると考えられる。
PS VRのコンテンツとして販売されている``サマーレッスン''というソフトがあるが、
これはVR空間での疑似恋愛を可能とするものである。
今後、AIなどの人工知能の発達とともに、疑似恋愛などの技術も発達していくと考えられる。
自分の好きなような、容姿、性格の相手が作れるというのは、真剣に取り組んでいかなければいけない問題ではなかろうか。



\begin{thebibliography}{99}
  \bibitem{youtube1} ``Making Virtual Reality World Record History'' \url{https://www.youtube.com/watch?v=mZau6PiLoJc} アクセス日 2017/01/16
  \bibitem{gigazine} ``ギネス世界記録となるVR空間の25時間ぶっ続け体験、果たしてプレイヤーに何が起こったのか?'' GIGAZINE \url{http://gigazine.net/news/20160512-virtual-reality-25-hours/} アクセス日 2017/01/16
\end{thebibliography}


\end{document}

\documentclass[11pt,a4j]{jsarticle}

\usepackage{float,array,booktabs,here}
\usepackage{amsmath}
\usepackage[dvipdfmx]{graphicx}
\usepackage[top=20truemm,bottom=25truemm,left=20truemm,right=20truemm]{geometry}
\usepackage{url}

\makeatletter
\newcommand{\figcaption}[1]{\def\@captype{figure}\caption{#1}}
\newcommand{\tblcaption}[1]{\def\@captype{table}\caption{#1}}
\makeatother

\newcommand{\Maru}[1]{\ooalign{
\ifnum#1<10 \hfil\resizebox{.9\width}{.85\height}{#1}\hfil
\else
\hfil\resizebox{.6\width}{.8\height}{#1}\hfil
\fi
\crcr
\raise.1ex\hbox{$\bigcirc$}}}


\begin{document}

% \title{フィルター}
% \author{hogehoge}
% \date{2015年5月9日}
% \maketitle


\section{目的}

光センサの一つである

\section{理論}
\label{sec:理論}







\section{実験方法}
\label{sec:実験方法}



\section{結果}
\label{sec:結果}




\section{考察}
\label{sec:考察}

\section{結論と感想}
\label{sec:結論と感想}


\begin{thebibliography}{9} %ケタ数(9:一桁、99:二桁)
\bibitem{saito} 斎藤英雄 , ``画像からの立体形状キャプチャ'' , 2016
\end{thebibliography}

\end{document}
